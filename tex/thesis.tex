%% Load document class fithesis2
%% {10pt, 11pt, 12pt}
%% {draft, final}
%% {oneside, twoside}
%% {onecolumn, twocolumn}
\documentclass[11pt,final,twoside]{fithesis2}

%% Basic packages
\usepackage[czech]{babel}
\usepackage{cmap}
\usepackage[T1]{fontenc}
\usepackage{lmodern}
\usepackage[utf8]{inputenc}
\usepackage{graphicx}

%% Additional packages for colors, advanced
%% formatting options, etc.
\usepackage{color}
\usepackage{microtype}
\usepackage{url}
\usepackage{cslatexquotes}
\usepackage{fancyvrb}
\usepackage[small,bf]{caption}
\usepackage[plainpages=false,pdfpagelabels,unicode]{hyperref}
\usepackage[all]{hypcap}

%% Fix long URLs in DVIs
\usepackage{ifpdf}

\ifpdf
\else
  \usepackage{breakurl}
\fi

%% Packages used to generate various lists
%\usepackage{makeidx}
%\makeindex

\usepackage[xindy]{glossaries}
\makeglossary

%% Use STAR and CIRCLE signs for nested
%% itemized lists
\renewcommand{\labelitemii}{$\star$}
\renewcommand{\labelitemiii}{$\circ$}

%% Title page information
\thesistitle{Komponenta distribuce a správy klíčů pro bezdrátové sensorové síťě}
\thesissubtitle{bakalářská práce}
\thesisstudent{Lukáš Němec}
\thesiswoman{false} %% Important when using Slovak or Czech lang
\thesisfaculty{fi}  %% {fi, eco, law, sci, fsps, phil, ped, med, fss}
\thesislang{cs}     %% {en, sk, cs}
\thesisyear{jaro 2014}
\thesisadvisor{RNDr. Petr Švenda, Ph.D.}

%% Beginning of the document
\begin{document}

%% Front page with a logo and basic thesis information
\FrontMatter
\ThesisTitlePage

%% Thesis declaration (required)
\begin{ThesisDeclaration}
  \DeclarationText
  \AdvisorName
\end{ThesisDeclaration}

%% Thanks (optional)
\begin{ThesisThanks}
poděkování
\end{ThesisThanks}

%% Abstract (required)
\begin{ThesisAbstract}
abstrakt
\end{ThesisAbstract}

%% Keywords (required)
\begin{ThesisKeyWords}
klíčová, slova
\end{ThesisKeyWords}

%% Beginning of the thesis itself
\MainMatter

%% TOC (required)
\tableofcontents

%% Thesis text structured using
%% chapters, sections, subsections, etc.
\chapter{Úvod}


TODO

\chapter{Existující implementace pro distribuci a správu klíčů}
Mechanismus bexpečného ustanovení klíčů pro komunikaci v bezdrátové sensorové síti tvoří jeden ze základních předpokladů 
pro bezpečné a spolehivé fungování sítě jako celku. Na zabezpečenou komunikaci například spoléhá ... TODO examples plus citations

Přitom co se možných řešení týká, jsme zde limitování mnohými rozdíly oproti sítím v klasické podobě. Mezi nejvýznamnější rozdíly například
patří omezené možnosti napájení (jednotlivé uzly mají většinou vlastní baterii a snahou je interval nutných výměn prodloužit co nejvíce), 
většinu spojení nemůžeme považovat za trvalé (topologie sítě se může měnit, případně uzly z důvodu úspory energie komunikují pouze v určitých 
intervalech) a musíme počítat s faktem, že hardware uzlu, především z důvodu ceny, nemá žádnou ochranu proti násilnému získání obsahu paměti uzlu, 
tedy i případných kryptogafických klíčů v paměti obsažených. Následují tedy jednotlivé klasické možnosti správy klíčů s odůvodněním proč nejsou pro 
použítí v rámci sensorové sítě vhodné.

\subsection{Jeden sdílený klíč (master key)} Řešení pomocí jednoho sdíleného klíče, který zabezpečuje vešekerou komunikaci je sice nejjednodušší a 
nejméně náročné co se implementace týká, nicméně tím výčet výhod tohoto řešení končí a následují nevýhodny, které převažují. Tou nejdůležitější 
je problém v případě, že se útočníkovi podaří získat jedinný uzel z celé sítě, tedy získá sdílený klíč pro komunikaci v rámci celé sítě. Následně 
má jednak možnost dešifrovat všechny zprávy zaslané v síti a jednak má možnost vysílat vlastní zprávy bez možnosti detekce.

\subsection{Párové klíče (Pairwise keys)} Zde již je kompromitování komunikace v celé síti vyřešeno pomocí separátních klíčů pro každý uzel, 
tedy útočník se získáním uzlu získá pouze zprávy směřující k tomuto uzlu, případně možnost vydávat se za tento uzel, nicméně toto řešení 
obsahuje jiný problém. V případě skutečně rozsáhlé sítě (tísíce až statisíce uzlů) není technicky možné, aby paměť uzlu obsahovala natolik 
velké množství klíčů, minimálně hardware který je v současné době dostupný pro stavbu sensorových sítí toto neumožnuje\footnote{pokud uvažujeme 
AES\cite{Daemen1999} klíč o velikosti 128 bitů a 1000 uložených klíčů v paměti uzlu, dostáváme celkem 16KB dat, které je potřeba uložit do paměti uzlu. 
Pokud uzažujeme například platformu TelosB\cite{MemsicInc.}, zde máme k dispozici právě 16KB EEPROM paměti určené k ukládání dat. Musíme však počítat s 
tím, že není primárním cílem zabezpečená komunikace, ale sběr dat pomocí senzorů, tedy kryptografická data by neměla zabírat většinu paměťového prostoru 
uzlu a to neuvažujme rozsáhlejší sítě s násobně větším množstvím uzlů}. Hlavní slabinou této varianty je tedy velmi velká paměťová náročnost a obtížná 
škálovatelnost pro rozsáhlejší sítě. 

\subsection{Asymetrická kryptografie} Při použití asymetrické kryptografie narážíme na problém ve omezeném výpočetním výkonu jednotlivých uzlů, 
které jsou aktuálně schopné provádět pouze podmnožinu operací potřebných k bezpečnému fungování asymetrické kryptografie tak jak ji známe.
Na druhou stranu existují implementace asymetrické kryptografie pro použití v rámci bezdrátových sensorových sítí \cite{Watro2004}, nicméně 
tyto nejsou určeny pro komunikaci mezi jednotlivými uzly navzájem, ale spíše pro zasílání informací od sensorů k třetí straně. 
Navíc při použití veřejných klíčů vyvstává problém s jejich revokací a opět je zde problém, že klíč z uzlu, který získá útočník může být použit 
na kterémkoliv místě v síti.

Celkově bráno, většina známých klasických přístupů není vhodná pro řešení správy klíčů v bezdrátové sensorové síti, je tedy na místě hledat alternativní řešení,
která již budou brát v úvahu všechna omezení těchto sítí a využijí vlastnosti bezdrátových sensorových sítí ve svůj prospěch.

\section{Pravděpodobnostní rozdělení klíčů (Eschenauer and Gligor scheme)}
Návrh pravděpodobnostního rozdělení klíčů\cite{Eschenauer2002} řeší elegantním způsobem problém přílišné paměťové náročnosti při použití párových klíčů a je 
založen na narozeninovém paradoxu\cite{Flajolet1992} díky kterému můžeme vypočítat pravděpodobnost s jakou dva uzly sdílejí stejný klíč. 
Celkově se schéma skládá ze tří fází, první obsahuje predistribuci vybraných klíčů, druhá nalezení sdílených klíčů a třetí řeší nalezení cest v síti.

\subsection{Popis fungování schématu}
Samotná první fáze se skládá z několika kroků, nejdříve je vygenerována velká množina $P$ klíčů, z těchto je následně náhodně vybráno $k$ klíčů pro každý
uzel. Počet $k$ vybraných klíčů je určen celkovým počtem uzlů a požadovanou pravděpodobností sdíleného klíče mezi sousedními uzly. Celá tato první fáze 
probíhá před samotným rozmístěním uzlů. Následuje druhá fáze, která nastává ve fázi inicializace sítě, kdy jednotlivé uzly zjistí své sousedy v dosahu 
rádiové komunikace a zjistí, se kterými sdílí klíč. Nejjednoduší cesta, jak tot provédst je, že každý uzel odešle všem ostatním ve svém okolí seznam jím 
vlastněných klíčů. Seznam identifikátorů jednotlivých klíčů není potřeba považovat za citlivou informaci, proto tento postup postačuje. V případě potřeby 
jej ale lze nahradit tak, že uzel odešle výzvy zašifrované pomocí všech klíčů, které vlastní. pokud se příjemnci podaří výzvu úspěšně dešifrovat, pak
uzly mají společný klíč. Následuje ustanovení cest v síti, kdy je vyhledána cesta mezi uzly sítě, které nevlastní společný klíč, nicméně je možné je propojit 
komunikací přes ostatní uzly, které klíč sdílejí. Touto cestou je následně vyměněn nový klíč pro přímé spojení. 

Při následné komunikaci je v případě potřeby možné celý proces nalezení sdílených klíčů a cest v síti spustit znovu, například v případě získání části klíčů útočníkem
je možné tyto klíče revokovat (zasláním zprávy z centrálního uzlu) a cesty ustanovit znovu bez použití těchto klíčů. Stejný proces může být inicializován i samotným uzlem
a to v případě potřeby použití nových klíčů pro spojení. 

\subsection{Vhodnost použití}
Celé schéma si při své jednoduchosti zachovává velmi dobrou funkčnost a v případě potřeby komunikace pouze mezi uzly samotnými se pravděpodobně jedná o nejvýhodnější
variatu řešení a to především pro případy, kdy je rozmístění jednotlivých uzlů náhodné a nemůžeme počítat s fizní topologií sítě. Schéma je taktéž velmi dobře kombinovatelné
s dalšími jednoduchými způsoby komunikace a tedy rozšířitělné o případnou další funkcionalitu. Z hlediska implementace a náročnosti na hardware jednotlivých uzlů schéma
splňuje všechny požadavky, nicméně schéma tak jak je nevyhovuje pro použítí v rámci vyvíjené komponenty především z důvodu nedostatečné funkcionality. Schéma sice zajistí ustanovení
a správu komunikace mezi uzly samotnými, ale neřeší jakékoliv návazné problémy, jako například hromadné posílání zpráv z centrálního uzlu a vyžadovalo by příliš mnoho modifikací oproti 
původnímu návrhu a proto nebylo vybráno.

\section{TinyPK}
TinyPK\cite{Watro2004} je návrhem způsobu, jak použít principy asymetrické kryptografie 
pro komunikaci mezi uzly sensorové sítě a třetí stranou. Cílem atorů je eliminovat problémy spojené s využíváním 
symetrikých klíčů, především zásadní otázku týkající se správy symetrikých klíčů. K tomuto využívají
implementaci RSA algoritmu\cite{Rivest1978}, samozřejmně s omezeními, které stanovuje hardware jednotlivých uzlů.

Návrh samotné komunikace s těmito omezeními samozřejmně počítá a proto na samotných uzlech jsou používány
pouze operace s veřejným klíčem. Ze stejných důvodů taktéž není použit žádný standart certifikátů, ale 
pouze podpisy jednotlivých klíčů a zpráv. S tímto souvisí problém s revokací klíčů, nicméně autoři se tímto nezabývají.
Dále tento návrh počítá s existencí certifikační autority, jejiž veřejný klíč je předdistribuován na samotné uzly. 

\subsection{Princip ustanovení komunikace}
Komunikační protokol začíná s předpokladem předdistribuovaného veřejného klíče certifikační autority na jednotlivé
uzly a existencí třetí strany se zájmem komunikovat se sensorovou sítí. Komunikace samotná je iniciována třetí
stranou, konkrétně spojením s certifikační autoritou a následným podepsáním veřejného klíče třetí strany (klíč podepisuje 
certifikační autorita po patřičném ověření identity třetí strany). Tento krok může proběhnout s předstihem oproti samotné 
komunikaci se sensorovou sítí a tedy není vyžadováno, aby certifikační autorita byla neustále dostupná. 

Nyní již k samotné komunikaci se sensorovou sítí. Tato je opět iniciována třetí stranou a to tím, že pošle jednotlivým uzlům 
výzvu. První část zprávy tvoří veřejný klíč třetí strany (který je podepsaný certifikační autoritou) a druhá část je podepsána 
soukromým klíčem třetí stray a je složená z časového razítka (timestamp) a kontrolního součtu celé zprávy. Po příjetí zprávy
uzel ověří platnost podpisu certifikační autority a hodnotu kontrolního součtu, v případě že se uzlu podaří tyto údaje ověřit, 
považuje třetí stranu za autentizovanou a vytvoří odpověd, která obsahuje symetrický klíč pro další komunikaci a ke správě připojí 
původní časové razítko. Tuto zprávu zašifruje obdrženým veřejným klíčem a odešle. Třetí strana následně ověří, že se obdržené 
časové razítko shoduje s původním zaslaným a v případě kladného výsledku srovnání začnš používat symetrický klíč pro 
komunikaci s uzlem. 

Protokol je dále možné rozšířit o autentizaci uzlu třetí straně. Tato probíhá za pomoci Diffie-Hellman protokolu\cite{Diffie1976}, 
kdy komunikující strany nejdříve ustanoví společný klíč a následně uzel pošle zprávu, z jejíhož obsahu může třetí strana oveřit,
že uzel vlastní privátní část Diffie-Hellman klíče. Dále pro účely ověření zpráva obsahuje  textový řetězec podepsaný 
certifikační autoritou. Tento řetězec obsahuje sériové číslo uzlu, datum výroby a případně další údaje potřebné pro identifikaci. 
Takto může třetí strana ověřit všechny potřebné údaje ke správné identifikaci uzlu, včetně správnosti podpisu certifikační autority, 
a to bez nutnosti provedení kompletní RSA šifry na uzlu samotném.

\subsection{Vhodnost použití}

Zde je nutné konstatovat, že TinyPK i přes svoji velmi zajímavou koncepci neřeší požadovaný problém a to komunikaci mezi jednotlivými
uzly navzájem. Navíc použití RSA algoritmu, byť ve velmi omezené podobě, přináší velmi výraznou dobu výpočtu na jednotlivých uzlech sítě. 
Autoři článku implementovali TinyPK pro platformu TinyOS, konkrétně uzly typu MICA2\cite{Inc.} a v závisloti na dílce klíče a velikosti modulu
trvaly jednotlivé operace RSA algoritmu řádově desitky sekund. Což velmi výrazně limituje schopnost uzlu řešit v mezidobí jakékoliv jiné úkoly, 
tedy opět se nejedná o vhodné řešení. 

Celkově je TinyPK zajímavým konceptem, nicméně nesplňuje požadavky kladené na vyvíjenou komponentu pro distribuci a správu klíčů, tedy není vhodný 
pro požití. Nicméně v případě nutnosti rozšíření komponenty o schopnost komunikovat s třetí stranou by je jednalo o velmi zajímavou variantu řešení problému.





%% Lists of tables and figures, glossary, etc.
%\printindex
%\printglossary
%\listoffigures
%\listoftables

%% Bibliography from lib.bib
\begingroup
\def\tmpchapter{0}
\renewcommand{\chaptername}{}
\renewcommand{\thechapter}{}
\addtocontents{toc}{\setcounter{tocdepth}{-1}}
\chapter{Literatura}
\renewcommand{\chapter}[2]{}% for other classes

\bibliographystyle{plain}
\bibliography{lib.bib}

\addtocontents{toc}{\setcounter{tocdepth}{2}}
\endgroup

%% Additional materials
\appendix

%% End of the whole document
\end{document}