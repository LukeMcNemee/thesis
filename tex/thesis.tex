%% Load document class fithesis2
%% {10pt, 11pt, 12pt}
%% {draft, final}
%% {oneside, twoside}
%% {onecolumn, twocolumn}
\documentclass[11pt,final,twoside]{fithesis2}

%% Basic packages
\usepackage[czech]{babel}
\usepackage{cmap}
\usepackage[T1]{fontenc}
\usepackage{lmodern}
\usepackage[utf8]{inputenc}
\usepackage{graphicx}

%% Additional packages for colors, advanced
%% formatting options, etc.
\usepackage{color}
\usepackage{microtype}
\usepackage{url}
\usepackage{cslatexquotes}
\usepackage{fancyvrb}
\usepackage[small,bf]{caption}
\usepackage[plainpages=false,pdfpagelabels,unicode]{hyperref}
\usepackage[all]{hypcap}

%% Fix long URLs in DVIs
\usepackage{ifpdf}

\ifpdf
\else
  \usepackage{breakurl}
\fi

%% Packages used to generate various lists
%\usepackage{makeidx}
%\makeindex

\usepackage[xindy]{glossaries}
\makeglossary

%% Use STAR and CIRCLE signs for nested
%% itemized lists
\renewcommand{\labelitemii}{$\star$}
\renewcommand{\labelitemiii}{$\circ$}

%% Title page information
\thesistitle{Komponenta distribuce a správy klíčů pro bezdrátové sensorové síťě}
\thesissubtitle{bakalářská práce}
\thesisstudent{Lukáš Němec}
\thesiswoman{false} %% Important when using Slovak or Czech lang
\thesisfaculty{fi}  %% {fi, eco, law, sci, fsps, phil, ped, med, fss}
\thesislang{cs}     %% {en, sk, cs}
\thesisyear{jaro 2014}
\thesisadvisor{RNDr. Petr Švenda, Ph.D.}

%% Beginning of the document
\begin{document}

%% Front page with a logo and basic thesis information
\FrontMatter
\ThesisTitlePage

%% Thesis declaration (required)
\begin{ThesisDeclaration}
  \DeclarationText
  \AdvisorName
\end{ThesisDeclaration}

%% Thanks (optional)
\begin{ThesisThanks}
poděkování
\end{ThesisThanks}

%% Abstract (required)
\begin{ThesisAbstract}
abstrakt
\end{ThesisAbstract}

%% Keywords (required)
\begin{ThesisKeyWords}
klíčová, slova
\end{ThesisKeyWords}

%% Beginning of the thesis itself
\MainMatter

%% TOC (required)
\tableofcontents

%% Thesis text structured using
%% chapters, sections, subsections, etc.
\chapter{Úvod}


TODO

\chapter{Existující implementace pro distribuci a správu klíčů}
Mechanismus bexpečného ustanovení klíčů pro komunikaci v bezdrátové sensorové síti tvoří jeden ze základních předpokladů 
pro bezpečné a spolehivé fungování sítě jako celku. Na zabezpečenou komunikaci například spoléhá ... TODO examples plus citations

Přitom co se možných řešení týká, jsme zde limitování mnohými rozdíly oproti sítím v klasické podobě. Mezi nejvýznamnější rozdíly například
patří omezené možnosti napájení (jednotlivé uzly mají většinou vlastní baterii a snahou je interval nutných výměn prodloužit co nejvíce), 
většinu spojení nemůžeme považovat za trvalé (topologie sítě se může měnit, případně uzly z důvodu úspory energie komunikují pouze v určitých 
intervalech) a musíme počítat s faktem, že hardware uzlu, především z důvodu ceny, nemá žádnou ochranu proti násilnému získání obsahu paměti uzlu, 
tedy i případných kryptogafických klíčů v paměti obsažených. Následují tedy jednotlivé klasické možnosti správy klíčů s odůvodněním proč nejsou pro 
použítí v rámci sensorové sítě vhodné.

\subsection{Jeden sdílený klíč (master key)} Řešení pomocí jednoho sdíleného klíče, který zabezpečuje vešekerou komunikaci je sice nejjednodušší a 
nejméně náročné co se implementace týká, nicméně tím výčet výhod tohoto řešení končí a následují nevýhodny, které převažují. Tou nejdůležitější 
je problém v případě, že se útočníkovi podaří získat jedinný uzel z celé sítě, tedy získá sdílený klíč pro komunikaci v rámci celé sítě. Následně 
má jednak možnost dešifrovat všechny zprávy zaslané v síti a jednak má možnost vysílat vlastní zprávy bez možnosti detekce.

\subsection{Párové klíče (Pairwise keys)} Zde již je kompromitování komunikace v celé síti vyřešeno pomocí separátních klíčů pro každý uzel, 
tedy útočník se získáním uzlu získá pouze zprávy směřující k tomuto uzlu, případně možnost vydávat se za tento uzel, nicméně toto řešení 
obsahuje jiný problém. V případě skutečně rozsáhlé sítě (tísíce až statisíce uzlů) není technicky možné, aby paměť uzlu obsahovala natolik 
velké množství klíčů, minimálně hardware který je v současné době dostupný pro stavbu sensorových sítí toto neumožnuje\footnote{pokud uvažujeme 
AES \cite{Daemen1999} klíč o velikosti 128 bitů a 1000 uložených klíčů v paměti uzlu, dostáváme celkem 16KB dat, které je potřeba uložit do paměti uzlu. 
Pokud uvažujeme například platformu TelosB \cite{MemsicInc.}, zde máme k dispozici právě 16KB EEPROM paměti určené k ukládání dat. Musíme však počítat s 
tím, že není primárním cílem zabezpečená komunikace, ale sběr dat pomocí senzorů, tedy kryptografická data by neměla zabírat většinu paměťového prostoru 
uzlu a to neuvažujme rozsáhlejší sítě s násobně větším množstvím uzlů}. Hlavní slabinou této varianty je tedy velmi velká paměťová náročnost a obtížná 
škálovatelnost pro rozsáhlejší sítě. 

\subsection{Asymetrická kryptografie} Při použití asymetrické kryptografie narážíme na problém ve omezeném výpočetním výkonu jednotlivých uzlů, 
které jsou aktuálně schopné provádět pouze podmnožinu operací potřebných k bezpečnému fungování asymetrické kryptografie tak jak ji známe.
Na druhou stranu existují implementace asymetrické kryptografie pro použití v rámci bezdrátových sensorových sítí \cite{Watro2004}, nicméně 
tyto nejsou určeny pro komunikaci mezi jednotlivými uzly navzájem, ale spíše pro zasílání informací od sensorů k třetí straně. 
Navíc při použití veřejných klíčů vyvstává problém s jejich revokací a opět je zde problém, že klíč z uzlu, který získá útočník může být použit 
na kterémkoliv místě v síti.

Celkově bráno, většina známých klasických přístupů není vhodná pro řešení správy klíčů v bezdrátové sensorové síti, je tedy na místě hledat alternativní řešení,
která již budou brát v úvahu všechna omezení těchto sítí a využijí vlastnosti bezdrátových sensorových sítí ve svůj prospěch.

\section{Pravděpodobnostní rozdělení klíčů (Eschenauer a Gligor)} \label{sec:Eschenauer}
Návrh pravděpodobnostního rozdělení klíčů \cite{Eschenauer2002} řeší elegantním způsobem problém přílišné paměťové náročnosti při použití párových klíčů a je 
založen na narozeninovém paradoxu díky kterému můžeme vypočítat pravděpodobnost s jakou dva uzly sdílejí stejný klíč. 
Celkově se schéma skládá ze tří fází, první obsahuje predistribuci vybraných klíčů, druhá nalezení sdílených klíčů a třetí řeší nalezení cest v síti.

\subsection{Popis fungování schématu}
Samotná první fáze se skládá z několika kroků, nejdříve je vygenerována velká množina $P$ klíčů, z těchto je následně náhodně vybráno $k$ klíčů pro každý
uzel. Počet $k$ vybraných klíčů je určen celkovým počtem uzlů a požadovanou pravděpodobností sdíleného klíče mezi sousedními uzly. Celá tato první fáze 
probíhá před samotným rozmístěním uzlů. Následuje druhá fáze, která nastává ve fázi inicializace sítě, kdy jednotlivé uzly zjistí své sousedy v dosahu 
rádiové komunikace a zjistí, se kterými sdílí klíč. Nejjednoduší cesta, jak toto provédst je, že každý uzel odešle všem ostatním ve svém okolí seznam jím 
vlastněných klíčů. Seznam identifikátorů jednotlivých klíčů není potřeba považovat za citlivou informaci, proto tento postup postačuje. V případě potřeby 
jej ale lze nahradit tak, že uzel odešle výzvy zašifrované pomocí všech klíčů, které vlastní. pokud se příjemnci podaří výzvu úspěšně dešifrovat, pak
uzly mají společný klíč. Následuje ustanovení cest v síti, kdy je vyhledána cesta mezi uzly sítě, které nevlastní společný klíč, nicméně je možné je propojit 
komunikací přes ostatní uzly, které klíč sdílejí. Touto cestou je následně vyměněn nový klíč pro přímé spojení. 

Při následné komunikaci je v případě potřeby možné celý proces nalezení sdílených klíčů a cest v síti spustit znovu, například v případě získání části klíčů útočníkem
je možné tyto klíče revokovat (zasláním zprávy z centrálního uzlu) a cesty ustanovit znovu bez použití těchto klíčů. Stejný proces může být inicializován i samotným uzlem
a to v případě potřeby použití nových klíčů pro spojení. 

\subsection{Vhodnost použití}
Celé schéma si při své jednoduchosti zachovává velmi dobrou funkčnost a v případě potřeby komunikace pouze mezi uzly samotnými se pravděpodobně jedná o nejvýhodnější
variatu řešení a to především pro případy, kdy je rozmístění jednotlivých uzlů náhodné a nemůžeme počítat s předem určenou topologií sítě. Schéma je taktéž velmi dobře kombinovatelné
s dalšími jednoduchými způsoby komunikace a tedy rozšířitělné o případnou další funkcionalitu. Z hlediska implementace a náročnosti na hardware jednotlivých uzlů schéma
splňuje všechny požadavky, nicméně schéma tak jak je nevyhovuje pro použítí v rámci vyvíjené komponenty především z důvodu nedostatečné funkcionality. Schéma sice zajistí ustanovení
a správu komunikace mezi uzly samotnými, ale neřeší jakékoliv návazné problémy, jako například hromadné posílání zpráv z centrálního uzlu a vyžadovalo by příliš mnoho modifikací oproti 
původnímu návrhu a proto nebylo vybráno.

\section{Chan, Perrig a Song} \label{sec:Chan}
Návrh dvou možnosti rozšíření \cite{Perrig2003} předchozího schématu \cite{Eschenauer2002} tak, aby bylo odolnější vůči možným útokům a aby síť měla možnost znovu zabezpečit spojení, jehož
klíč byl prozrazen. Obě části počítají se stejnými podmínkami jako předchozí, tedy nahrání množiny klíčů na jednotlivé uzly s tím, že jednotlivé klíče jsou náhodně vybrány 
z větší množiny všech generovaných klíčů.

\subsection{q-kompozitní klíč (q-composite key)}
Oproti původnímu návrhu, zde je ustanovena hodnota $q$, která vyjadřuje počet sdílených klíčů, které musí sousední uzly sdílet, aby bylo možné ustanovit spojení. Při původní variantě 
je hodnota $q$ rovna jedné a tedy není potřeba se jí zabývat. Zde ovšem hraje důležitou roli z hlediska zvýšení zabezpečení sítě. Sousední uzly opět vysíláním identifikátorů klíčů které vlastní
zjistí, se kterými sousedy sdílejí alespoň $q$ klíčů a s těmito zahájí komunikaci. Předpokládejme tedy, že dva uzly sdílejí $q'$ klíčů $k_1, k_2 \dots k_{q'}$ ($q' \ge q$) a existenci hashovací funkce $H$, 
společný klíč $k$ pro následující komunikaci mezi uzly je tedy ustanoven následovně $k=H(k_1 || k_2 || \dots || k_{q'})$ s tím, že je zabezpečené stejné pořadí klíčů, například podle původního
pořadí v jakém byly generovany. 

Schéma zachovává původní princip nalezení cesty mezi uzly, které na začátku nevlastní potřebný počet společných klíčů a taktéž původní princip revokace klíčů pomocí zaslání zprávy z 
centrálního uzlu. Celkově zlepšení poskytuje zvýšenou ochranu proti útokům malého rozsahu za cenu snížení odolnosti proti útokům většího rozsahu. Zde je možné konstatovat, že se jedná 
o žádoucí vlastnosti, jelikož malé útoky je těžké zaznamenat (malé procento nefunkčních uzlů může být běžný jev), zatímco útoky na výrazně větší množství uzlů již pravděpodobně zaznamenány budou a 
bude tedy možné provédst patřičná protiopatření. 

\subsection{vícecestné obnovením spojení (multipath)}
Druhý návrh se zabývá obnovením spojení, jehož klíč byl prozrazen, nicméně uzly samotné zůstaly útokem nedotčeny. K této situaci například dojde, pokud je klíč používaný pro komunikaci mezi
uzly $A$ a $B$ uložen v paměti uzlu $C$ a útočník získá klíče na uložené na uzlu $C$. Obnovaní komunikačního klíče je provedeno přes více nezávislých cest, konkrétně uzly $A$ a $B$ zjistí společné sousedy
přes které se mohou spojit tak, že délka cesty je kratší než předem stanovená maximální hodnota přeskoků. Pokud tedy existuje $j$ cest spojujících tyto dva uzly, pka uzel $A$ pošle po mkaždé cestě náhodně 
generovanou hodnotu $v$, celkově tedy pošle hodnoty $v_1, v_2, \dots v_j$. Na uzlu $B$ je nová hondota klíče spočítána z původní hodnoty následovně $k' = k \oplus v_1 \oplus v_2 \oplus \dots \oplus v_j$

Tato metoda celkově způsobí znatelné zatížení sítě, především pro větší dovolený počet přeskoků. Pokud však dovolíme pouze malý počet přeskoků (například dva) zaznamenáme zlepšení v odolnosti proti 
případným útokům. Především síť dokáže obnovit spojení, která jsou přerušena vyzrazením použitého klíče.

\subsection{Vhodnost použití}
Autoři nedoporučují oba navržené přístupy kombinovat, protože při použití obou dojde ke zkombinovaní negativních dopadů obou, tedy zmenšení celkového počtu klíčů aby bylo dostatečné množstvé shodných
klíčů mezi jendotlivými uzly a k nárustu komunikace z důvodu obnovování jednotlivých spojení. 

Oba návrhy jsou zajímavým řešením, které zlepší zabezpečení sítě oproti původní variantě pravděpodobnostního rozdělení klíčů. Variaty se liší způsobem, kterým tohoto zpešení dosáhnou, q-composite 
varianta posiluje bezpečnost před samotným útokem a tedy odolnost vůči útoku, zatímco multipath poskytuje řešení pro případ úspěšného útoku a nabízí možnost jak minimalizovat škody útokem způsobené.
Co se vhodnoti použití na vyvíjené komponentě týká, opět jsou splněny požadavky kladené omezeným výkonem jednotlivých uzlů, nicméně i přes přidanou funkcionalitu (především ve 
variantě multipath) neobsahuje schéma všechny požadované vlastnosti. Především se nezabývá zabezpečením oboustrané komunikace s centrálním uzlem. Tedy použití by bylo vhodné pouze v kombinaci 
s jiným přístupem. Z tohoto důvodu tedy schéma nebylo vybráno. 


\section{Du, Deng, Han and Varshney}
Zajímavý návrh \cite{Du2005} koncepčně vycházející ze schématu původně navrženého Blomem \cite{Blom1985}, ale rozšiřující jej o pravděpodobnostní rozdělení klíčů \ref{sec:Eschenauer}.
Díky tomu získáváme schéma pro distribuci klíčů vhodné pro použítí v bezdrátových sensorových sítích. Oproti předchozím variantám návrh celkově zvyšuje odolnost sítě vůči 
útočníkovi a to za cenu vyšší výpočetní náročnosti, nicméně tato zůstává na rozumné míře (tedy stále výrazně menší než u asymetrické kryptografie). 

\subsection{původní Blomovo schéma}
Blom \cite{Blom1985} se zabývá řešením problému distribuce klíčů v klasické síti, především s ohledem na velký počet uživatelů a tedy nepraktičnost případně až nemožnost ukládat všechny klíče u uživatele.
Namísto toho navrhuje doručit každému uživateli relativně malé množství dat ze kterého dokáže všechny potřebné klíče odvodit. Schéma je kontruováno tak, aby každá dvojice uživatelů z celkového počtu $n$
dokázala odvodit společný klíč a to pouze při uložení $\lambda + 1$ klíčů, kde $\lambda \ll n$. Co se bezpečnosti týká, dokud není počet kompromitovaných uživatelů (jejich klíčů) vyšší než hodnota 
$\lambda$, pak můžeme všechna spojení považovat za bezpečná. V okamžiku kdy je hodnota $\lambda$ překročena, získává útočník přístup k veškeré komunikaci v síti. 

Schéma je založené na násobení matic, konkrétně proces ustanovení klíčů probíhá následovně: 
nejdříve je zkonstruována matice $G$ o rozměru $(\lambda + 1) \times n$ nad konečným tělesem $GF(q)$, kde sloupce matice $G$ jsou lineárně nezávislé a $q > n$.
Tato matice je veřejně známou informací a každý účastník, včetně útočníka, může znát její obsah. Následuje zkonstruování symetrické matice $D$ s rozměrem $(\lambda + 1) \times (\lambda + 1) $, 
opět nad konečným tělesem $GF(q)$. Narozdíl od předchozí matice, matice $D$ je tajnou informací a celý její obsah zná pouze strana zabezpečující její generování a distribuci jednotlivých částí 
účastníkům. Poté je spočítána matice $A= (DG)^T$ a každému účastníkovi je distribuován k-tý sloupec matice $G$ a k-tý řádek matice $A$ ($k \in \{1, \dots , n\}$). Následně komunikující strany
mohou spočítat klíče $k_{ij}$ a $k_{ji}$.  

\subsection{rozříření o pravděpodobnostní rozdělení}
Při vytváření spojení v bezdrátové sítí není zcela nezbytné, aby byl kterýkoliv uzel schopen spojení se všemi ostatními uzly, mnohdy to ani není možné z důvodu krátkého dosahu rádia. Tedy
graf spojení mezi jednotlivými uzly, který se snažíme vytvořit zcela jistě nemusí být úplný, stačí pokud se nám podaří vytvořit souvislý graf. Protože nepotřebujeme aby každé dva uzly 
měly možnost ustanovení společného klíče, můžeme si dovolit upravit původní Blomovo schéma pomocí myšlenky pravděpodobnostního rozdělení a díky tomu získat větší odolnost proti útokům. 

Pokud tedy původní schéma má jeden klíčový prostor (za klíčový prostor můžeme považovat dvojici matic $(D, G)$), v této variantě máme $\omega$ klíčových prostorů. Vycházíme z jedné společné matice $G$, 
která je kostruována tak, že je vybrán generátor $p$ z konečného tělesa $GF(q)$, kde $q$ je větší jak požadovaná délka klíče (a také $q > N$) a tento generátor následně jednoznačně určuje tvar matice $G$.
Dále konstruujeme $\omega$ symetrických matic $D$ a pro každý uzel jich vybereme $\tau$ ($\tau < \omega$). Následně proběhne iniciální fáze, kdy rozmístěné uzly zjistí spoelčné klíče se svými sousedy
(společný klíč lze mezi uzly ustanovit, pokud alespoň jeden z $\tau$ klíčových prostorů na uzlech je shodný). Následně po vytvoření počátečního spojeného grafu, za použití předdistribuovaných hodnot, 
může případně probehnout následující fáze, kdy dojde k ustanovení spojení mezi sousedy, kterým se nepodařilo spojení navázat ve fázi předchozí. Zde tyto dvojice využijí již ustanovené cesty, po kterých si 
vymnění informace potřebné k navázání bezpečné komunikace. 

\subsection{Vhodnost použití}
Toto schéma nám při použití skýtá několik zajímavých výhod, stejně jako předchozí varianty je velmi dobře škálovatelné a to i pro opravdu velké množství uzlů\footnote{pokud budeme uvažovat 64bitový klíč, 
pak schéma umožní použít až $2^{64}$ uzlů}. Schéma má taktéž výrazně větší odolnost vůči útoku, oproti variantám \ref{sec:Eschenauer} nebo \ref{sec:Chan} musí útočník kompromitovat až pětinásobek počtu uzlů, 
aby dosáhnul zaručeného úspěchu při napadení sítě. Taktéž z hlediska paměťové náročnosti lze schéma považovat za vyhovující, protože pro uložení matice $G$ potřebujeme uložit pouze jeden její prvek 
(matice je jednoznačně určená svým generátorem) potřebujeme na každém uzlu uložit $\tau \cdot (\lambda +1)$ položek klíčů, což je výrazně méně než varianta uložení všech vzájemných klíčů. Navíc 
poměr mezi bezpečností a náročností na provoz můžeme upravovat vhodným nastavením hodnot jak parametru $\lambda$, tak i parametru $\tau$. Oba parametry ovlivňují paměťovou náročnost, čím větší hodnota, tím 
více informací je třeba uložit. Dále pak $\tau$ ovlivňuje pravděpodobnost, že dva uzly naleznou společný klíč, cíž vyšší hodnota $\tau$ je, tím větší je pravděpodobnost nalezení shodného klíčového prostoru
dvou sousedících uzlů. Hodnota $\lambda$ pak ovlivňuje bezpečnost sítě, protože útočník musí kompromitovat alespoň $\lambda$ uzlů, aby byl schopen kompromitovat celou síť, tedy čím vyšší hodnota, tím 
je síť odolnější proti potenciálnímu útoku.

Schéma je sice co se implementace a paměti týká náročnější než předchozí varinaty, nicméně tato náročnost přináší výhody v oblasti bezpečnosti a to výhody nikoliv 
nepodstatné. Jedná se tedy o pravděpodobně nejslibnější variatu pravděpodobnostního rozdělení klíčů ze všech tří diskutovaných. Nicméně stejně jako předchozí varinaty, toto schéma nepokrývá celý 
zamýšlený rozsah vyvíjené komponenty a proto by bylo potřeba jej rozšířit či kombinovat s jiným schématem. Navíc schéma počítá s náhodným rozmístěním jendotlivých uzlů v prostoru, zatímco v našem případě 
bude poloha uzlů předem známa, tedy potenciál tohoto schématu by nebyl zcela využit. Z těchto důvodů tedy schéma nebylo vybráno.





\section{TinyPK}
TinyPK \cite{Watro2004} je návrhem způsobu, jak použít principy asymetrické kryptografie 
pro komunikaci mezi uzly sensorové sítě a třetí stranou. Tedy toto schéma by mohlo být využito k rozšíření předchozích variant, například \ref{sec:Eschenauer}
nebo dalších. Cílem atorů je eliminovat problémy spojené s využíváním 
symetrikých klíčů, především zásadní otázku týkající se správy symetrikých klíčů. K tomuto využívají
implementaci RSA algoritmu \cite{Rivest1978}, samozřejmně s omezeními, které stanovuje hardware jednotlivých uzlů.

Návrh samotné komunikace s těmito omezeními samozřejmně počítá a proto na samotných uzlech jsou používány
pouze operace s veřejným klíčem. Ze stejných důvodů taktéž není použit žádný standart certifikátů, ale 
pouze podpisy jednotlivých klíčů a zpráv. S tímto souvisí problém s revokací klíčů, nicméně autoři se tímto nezabývají.
Dále tento návrh počítá s existencí certifikační autority, jejiž veřejný klíč je předdistribuován na samotné uzly. 

\subsection{Princip ustanovení komunikace}
Komunikační protokol začíná s předpokladem předdistribuovaného veřejného klíče certifikační autority na jednotlivé
uzly a existencí třetí strany se zájmem komunikovat se sensorovou sítí. Komunikace samotná je iniciována třetí
stranou, konkrétně spojením s certifikační autoritou a následným podepsáním veřejného klíče třetí strany (klíč podepisuje 
certifikační autorita po patřičném ověření identity třetí strany). Tento krok může proběhnout s předstihem oproti samotné 
komunikaci se sensorovou sítí a tedy není vyžadováno, aby certifikační autorita byla neustále dostupná. 

Nyní již k samotné komunikaci se sensorovou sítí. Tato je opět iniciována třetí stranou a to tím, že pošle jednotlivým uzlům 
výzvu. První část zprávy tvoří veřejný klíč třetí strany (který je podepsaný certifikační autoritou) a druhá část je podepsána 
soukromým klíčem třetí stray a je složená z časového razítka (timestamp) a kontrolního součtu celé zprávy. Po příjetí zprávy
uzel ověří platnost podpisu certifikační autority a hodnotu kontrolního součtu, v případě že se uzlu podaří tyto údaje ověřit, 
považuje třetí stranu za autentizovanou a vytvoří odpověd, která obsahuje symetrický klíč pro další komunikaci a ke správě připojí 
původní časové razítko. Tuto zprávu zašifruje obdrženým veřejným klíčem a odešle. Třetí strana následně ověří, že se obdržené 
časové razítko shoduje s původním zaslaným a v případě kladného výsledku srovnání začnš používat symetrický klíč pro 
komunikaci s uzlem.

Protokol je dále možné rozšířit o autentizaci uzlu třetí straně. Tato probíhá za pomoci Diffie-Hellman protokolu \cite{Diffie1976}, 
kdy komunikující strany nejdříve ustanoví společný klíč a následně uzel pošle zprávu, z jejíhož obsahu může třetí strana oveřit,
že uzel vlastní privátní část Diffie-Hellman klíče. Dále pro účely ověření zpráva obsahuje  textový řetězec podepsaný 
certifikační autoritou. Tento řetězec obsahuje sériové číslo uzlu, datum výroby a případně další údaje potřebné pro identifikaci. 
Takto může třetí strana ověřit všechny potřebné údaje ke správné identifikaci uzlu, včetně správnosti podpisu certifikační autority, 
a to bez nutnosti provedení kompletní RSA šifry na uzlu samotném.

\subsection{Vhodnost použití}

Zde je nutné konstatovat, že TinyPK neřeší komunikaci mezi jednotlivými uzly navzájem, ale komunikaci se třetí stranou. V případě vyvíjené 
komponenty lze tento přístup aplikovat na komunikaci s centrálním uzlem. Tedy TinyPK by byl ideálním kandidátem na rozšíření schématu pravděpodobnostního
rozdělení klíčů \ref{sec:Eschenauer}, případně ostatních schémat, které pravděpodobnostní rozdělení klíčů rozšiřují.
Bohužel použití RSA algoritmu, byť ve velmi omezené podobě, přináší relativně dlouhou dobu výpočtu na jednotlivých uzlech sítě\footnote{ 
Autoři článku implementovali TinyPK pro platformu TinyOS, konkrétně uzly typu MICA2 \cite{Inc.} a v závisloti na dílce klíče a velikosti modulu
trvaly jednotlivé operace RSA algoritmu řádově desítky sekund.} Jedná se sice pouze o počáteční fázi ustanovení sdíleného klíče, nicméně zpoždění je znatelné.

Výhodou použití tohoto schématu je, že v době rozmístění uzlů nemusí být třetí strana známa a díky použití asymetrické kryptografie může navázat komunikaci kdykoliv.
Nicméně v případě vyvíjené komponenty toto není případ, který by bylo potřeba řešit. Třetí strana v podobě centrálního uzlu sice existuje, nicméně je v síti přítomná již 
od začátku a se sítí komunikuje neustále. Implementovat další šifrovací algoritmus, který nám posluží pouze k počátečnímu ustanovení komunikace, když stejně dobře 
nám může posloužit jiý implementovaná funkcionalita je plýtvání dostupnými zdroji, které jsou omezené. TinyPK najde využití v případech, kdy síť potřebuje komunikovat
se třetí stanou po omezenou dobu, s tím že třetí strana není přítomna od začátku existence sítě a po získání potřebných informací komunikaci ukončí. Zde můžeme konstatovat, že 
TinyPK nezapadá do celkového konceptu zamýšlené funkcionality vyvíjené komponenty a proto toto schéma nebylo vybráno. 




%% Lists of tables and figures, glossary, etc.
%\printindex
%\printglossary
%\listoffigures
%\listoftables

%% Bibliography from lib.bib
\begingroup
\def\tmpchapter{0}
\renewcommand{\chaptername}{}
\renewcommand{\thechapter}{}
\addtocontents{toc}{\setcounter{tocdepth}{-1}}
\chapter{Literatura}
\renewcommand{\chapter}[2]{}% for other classes

\bibliographystyle{plain}
\bibliography{lib.bib}

\addtocontents{toc}{\setcounter{tocdepth}{2}}
\endgroup

%% Additional materials
\appendix

%% End of the whole document
\end{document}